\problemname{Chokladkartongen}
Bosse tycker om choklad.
Han har därför alltid en öppnad chokladkartong i skafferiet.
När den tar slut köper han i hemlighet en ny och låtsas som ingenting.
Bosses fru, som är misstänksam av naturen, förundras över att den där kartongen aldrig tar slut.
Därför börjar hon då och då räkna antalet chokladbitar som är kvar.
Skriv ett program som, givet hennes observationer, beräknar det minsta antalet nya kartonger Bosse kan ha köpt under perioden.

\section*{Indata}
På första raden står ett heltal $N \le 100$, antal observationer.
Därefter följer en rad med $N$ heltal, antalet chokladbitar i asken (mellan $1$ och $100$) vid varje observation, i den ordning de görs.

\section*{Utdata}
Programmet ska skriva ut en rad med ett heltal: det minsta antal nya kartonger Bosse bevisligen måste ha köpt under perioden.

\section*{Poängsättning}
Din lösning kommer att testas på flera testfall. För att få 100 poäng så måste du klara alla testfall.
